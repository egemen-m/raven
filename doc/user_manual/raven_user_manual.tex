%
% This is an example LaTeX file which uses the SANDreport class file.
% It shows how a SAND report should be formatted, what sections and
% elements it should contain, and how to use the SANDreport class.
% It uses the LaTeX article class, but not the strict option.
% ItINLreport uses .eps logos and files to show how pdflatex can be used
%
% Get the latest version of the class file and more at
%    http://www.cs.sandia.gov/~rolf/SANDreport
%
% This file and the SANDreport.cls file are based on information
% contained in "Guide to Preparing {SAND} Reports", Sand98-0730, edited
% by Tamara K. Locke, and the newer "Guide to Preparing SAND Reports and
% Other Communication Products", SAND2002-2068P.
% Please send corrections and suggestions for improvements to
% Rolf Riesen, Org. 9223, MS 1110, rolf@cs.sandia.gov
%
\documentclass[pdf,12pt]{INLreport}
% pslatex is really old (1994).  It attempts to merge the times and mathptm packages.
% My opinion is that it produces a really bad looking math font.  So why are we using it?
% If you just want to change the text font, you should just \usepackage{times}.
% \usepackage{pslatex}
\usepackage{times}
\usepackage{longtable}
\usepackage[FIGBOTCAP,normal,bf,tight]{subfigure}
\usepackage{amsmath}
\usepackage{amssymb}
\usepackage[labelfont=bf]{caption}
\usepackage{pifont}
\usepackage{enumerate}
\usepackage{listings}
\usepackage{fullpage}
\usepackage{xcolor}          % Using xcolor for more robust color specification
\usepackage{ifthen}          % For simple checking in newcommand blocks
\usepackage{textcomp}
%\usepackage{authblk}         % For making the author list look prettier
%\renewcommand\Authsep{,~\,}

% Custom colors
\definecolor{deepblue}{rgb}{0,0,0.5}
\definecolor{deepred}{rgb}{0.6,0,0}
\definecolor{deepgreen}{rgb}{0,0.5,0}
\definecolor{forestgreen}{RGB}{34,139,34}
\definecolor{orangered}{RGB}{239,134,64}
\definecolor{darkblue}{rgb}{0.0,0.0,0.6}
\definecolor{gray}{rgb}{0.4,0.4,0.4}

\lstset {
  basicstyle=\ttfamily,
  frame=single
}

\setcounter{secnumdepth}{5}
\lstdefinestyle{XML} {
    language=XML,
    extendedchars=true,
    breaklines=true,
    breakatwhitespace=true,
%    emph={name,dim,interactive,overwrite},
    emphstyle=\color{red},
    basicstyle=\ttfamily,
%    columns=fullflexible,
    commentstyle=\color{gray}\upshape,
    morestring=[b]",
    morecomment=[s]{<?}{?>},
    morecomment=[s][\color{forestgreen}]{<!--}{-->},
    keywordstyle=\color{cyan},
    stringstyle=\ttfamily\color{black},
    tagstyle=\color{darkblue}\bf\ttfamily,
    morekeywords={name,type},
%    morekeywords={name,attribute,source,variables,version,type,release,x,z,y,xlabel,ylabel,how,text,param1,param2,color,label},
}
\lstset{language=python,upquote=true}

\usepackage{titlesec}
\newcommand{\sectionbreak}{\clearpage}
\setcounter{secnumdepth}{4}

%\titleformat{\paragraph}
%{\normalfont\normalsize\bfseries}{\theparagraph}{1em}{}
%\titlespacing*{\paragraph}
%{0pt}{3.25ex plus 1ex minus .2ex}{1.5ex plus .2ex}

%%%%%%%% Begin comands definition to input python code into document
\usepackage[utf8]{inputenc}

% Default fixed font does not support bold face
\DeclareFixedFont{\ttb}{T1}{txtt}{bx}{n}{9} % for bold
\DeclareFixedFont{\ttm}{T1}{txtt}{m}{n}{9}  % for normal

\usepackage{listings}

% the following allows for 8 levels list nesting
\usepackage{enumitem}
\setlistdepth{8}
\setlist[itemize,1]{label=$\bullet$}
\setlist[itemize,2]{label=$\bullet$}
\setlist[itemize,3]{label=$\bullet$}
\setlist[itemize,4]{label=$\bullet$}
\setlist[itemize,5]{label=$\bullet$}
\setlist[itemize,6]{label=$\bullet$}
\setlist[itemize,7]{label=$\bullet$}
\setlist[itemize,8]{label=$\bullet$}
\renewlist{itemize}{itemize}{8}

% Python style for highlighting
\newcommand\pythonstyle{\lstset{
language=Python,
basicstyle=\ttm,
otherkeywords={self, none, return},             % Add keywords here
keywordstyle=\ttb\color{deepblue},
emph={MyClass,__init__},          % Custom highlighting
emphstyle=\ttb\color{deepred},    % Custom highlighting style
stringstyle=\color{deepgreen},
frame=tb,                         % Any extra options here
showstringspaces=false            %
}}


% Python environment
\lstnewenvironment{python}[1][]
{
\pythonstyle
\lstset{#1}
}
{}

% Python for external files
\newcommand\pythonexternal[2][]{{
\pythonstyle
\lstinputlisting[#1]{#2}}}

\lstnewenvironment{xml}
{}
{}

% Python for inline
\newcommand\pythoninline[1]{{\pythonstyle\lstinline!#1!}}

% Named Colors for the comments below (Attempted to match git symbol colors)
\definecolor{RScolor}{HTML}{8EB361}  % Sonat (adjusted for clarity)
\definecolor{DPMcolor}{HTML}{E28B8D} % Dan
\definecolor{JCcolor}{HTML}{82A8D9}  % Josh (adjusted for clarity)
\definecolor{AAcolor}{HTML}{8D7F44}  % Andrea
\definecolor{CRcolor}{HTML}{AC39CE}  % Cristian
\definecolor{RKcolor}{HTML}{3ECC8D}  % Bob (adjusted for clarity)
\definecolor{DMcolor}{HTML}{276605}  % Diego (adjusted for clarity)
\definecolor{PTcolor}{HTML}{990000}  % Paul

%\def\DRAFT{} % Uncomment this if you want to see the notes people have been adding
% Comment command for developers (Should only be used under active development)
\ifdefined\DRAFT
  \newcommand{\nameLabeler}[3]{\textcolor{#2}{[[#1: #3]]}}
\else
  \newcommand{\nameLabeler}[3]{}
\fi
\newcommand{\alfoa}[1] {\nameLabeler{Andrea}{AAcolor}{#1}}
\newcommand{\cristr}[1] {\nameLabeler{Cristian}{CRcolor}{#1}}
\newcommand{\mandd}[1] {\nameLabeler{Diego}{DMcolor}{#1}}
\newcommand{\maljdan}[1] {\nameLabeler{Dan}{DPMcolor}{#1}}
\newcommand{\cogljj}[1] {\nameLabeler{Josh}{JCcolor}{#1}}
\newcommand{\bobk}[1] {\nameLabeler{Bob}{RKcolor}{#1}}
\newcommand{\senrs}[1] {\nameLabeler{Sonat}{RScolor}{#1}}
\newcommand{\talbpaul}[1] {\nameLabeler{Paul}{PTcolor}{#1}}
% Commands for making the LaTeX a bit more uniform and cleaner
\newcommand{\TODO}[1]    {\textcolor{red}{\textit{(#1)}}}
\newcommand{\xmlAttrRequired}[1] {\textcolor{red}{\textbf{\texttt{#1}}}}
\newcommand{\xmlAttr}[1] {\textcolor{cyan}{\textbf{\texttt{#1}}}}
\newcommand{\xmlNodeRequired}[1] {\textcolor{deepblue}{\textbf{\texttt{<#1>}}}}
\newcommand{\xmlNode}[1] {\textcolor{darkblue}{\textbf{\texttt{<#1>}}}}
\newcommand{\xmlString}[1] {\textcolor{black}{\textbf{\texttt{'#1'}}}}
\newcommand{\xmlDesc}[1] {\textbf{\textit{#1}}} % Maybe a misnomer, but I am
                                                % using this to detail the data
                                                % type and necessity of an XML
                                                % node or attribute,
                                                % xmlDesc = XML description
\newcommand{\default}[1]{~\\*\textit{Default: #1}}
\newcommand{\nb} {\textcolor{deepgreen}{\textbf{~Note:}}~}

%%%%%%%% End comands definition to input python code into document

%\usepackage[dvips,light,first,bottomafter]{draftcopy}
%\draftcopyName{Sample, contains no OUO}{70}
%\draftcopyName{Draft}{300}

% The bm package provides \bm for bold math fonts.  Apparently
% \boldsymbol, which I used to always use, is now considered
% obsolete.  Also, \boldsymbol doesn't even seem to work with
% the fonts used in this particular document...
\usepackage{bm}

% Define tensors to be in bold math font.
\newcommand{\tensor}[1]{{\bm{#1}}}

% Override the formatting used by \vec.  Instead of a little arrow
% over the letter, this creates a bold character.
\renewcommand{\vec}{\bm}

% Define unit vector notation.  If you don't override the
% behavior of \vec, you probably want to use the second one.
\newcommand{\unit}[1]{\hat{\bm{#1}}}
% \newcommand{\unit}[1]{\hat{#1}}

% Use this to refer to a single component of a unit vector.
\newcommand{\scalarunit}[1]{\hat{#1}}

% \toprule, \midrule, \bottomrule for tables
\usepackage{booktabs}

% \llbracket, \rrbracket
\usepackage{stmaryrd}

\usepackage{hyperref}
\hypersetup{
    colorlinks,
    citecolor=black,
    filecolor=black,
    linkcolor=black,
    urlcolor=black
}

\newcommand{\wiki}{\href{https://github.com/idaholab/raven/wiki}{RAVEN wiki}}

% Compress lists of citations like [33,34,35,36,37] to [33-37]
\usepackage{cite}

% If you want to relax some of the SAND98-0730 requirements, use the "relax"
% option. It adds spaces and boldface in the table of contents, and does not
% force the page layout sizes.
% e.g. \documentclass[relax,12pt]{SANDreport}
%
% You can also use the "strict" option, which applies even more of the
% SAND98-0730 guidelines. It gets rid of section numbers which are often
% useful; e.g. \documentclass[strict]{SANDreport}

% The INLreport class uses \flushbottom formatting by default (since
% it's intended to be two-sided document).  \flushbottom causes
% additional space to be inserted both before and after paragraphs so
% that no matter how much text is actually available, it fills up the
% page from top to bottom.  My feeling is that \raggedbottom looks much
% better, primarily because most people will view the report
% electronically and not in a two-sided printed format where some argue
% \raggedbottom looks worse.  If we really want to have the original
% behavior, we can comment out this line...
\raggedbottom
\setcounter{secnumdepth}{5} % show 5 levels of subsection
\setcounter{tocdepth}{5} % include 5 levels of subsection in table of contents

% ---------------------------------------------------------------------------- %
%
% Set the title, author, and date
%
\title{RAVEN User Manual}
%\author{%
%\begin{tabular}{c} Author 1 \\ University1 \\ Mail1 \\ \\
%Author 3 \\ University3 \\ Mail3 \end{tabular} \and
%\begin{tabular}{c} Author 2 \\ University2 \\ Mail2 \\ \\
%Author 4 \\ University4 \\ Mail4\\
%\end{tabular} }


\author{
\textbf{\textit{Project Manager:}}
 \\Diego Mandelli\\
 \textbf{\textit{Principal Investigator and Technical Leader:}}
 \\Congjian Wang\\
\textbf{\textit{Main Developers:}}
\\Andrea Alfonsi
\\Diego Mandelli
\\Joshua Cogliati
\\Congjian Wang
\\Paul W. Talbot
\\Mohammad G. Abdo
\\Dylan J. McDowell
\\Ramon K. Yoshiura
\\Junyung Kim
\\Gabriel J. Soto\\
\textbf{\textit{Former Developers:}}
\\Cristian Rabiti
\\Daniel P. Maljovec
\\Sonat Sen
\\Robert Kinoshita
\\Jun Chen
\\Jia Zhou
\\Daniel Garrett\\
\textbf{\textit{Contributors:}}
\\Alessandro Bandini (Post-Processor)
\\Ivan Rinaldi (documentation)
\\Claudia Picoco (new external code interface)
\\James B. Tompkins (new external code interface)
\\Matteo Donorio (new external code interface)
\\Fabio Giannetti (new external code interface)
\\Alp Tezbasaran (new external code interface)
\\Anthoney A. Griffith (Bayesian optimization)
\\Jacob A. Bryan (TSA module)
\\Haoyu Wang (DMDc)
\\Khang Nguyen (new external code interface)
}

% There is a "Printed" date on the title page of a SAND report, so
% the generic \date should [WorkingDir:]generally be empty.
\date{}


% ---------------------------------------------------------------------------- %
% Set some things we need for SAND reports. These are mandatory
%
\SANDnum{INL/EXT-15-34123}
\SANDprintDate{\today}
\SANDauthor{Cristian Rabiti, Andrea Alfonsi, Joshua Cogliati, Diego Mandelli, Congjian Wang, Paul W. Talbot,
Mohammad G. Abdo, Dylan J. McDowell, Ramon K. Yoshiura, Daniel P. Maljovec, Jun Chen, Jia Zhou, Junyung Kim, Robert Kinoshita, Sonat Sen, Gabriel J. Soto}
\SANDreleaseType{Revision 10}

% ---------------------------------------------------------------------------- %
% Include the markings required for your SAND report. The default is "Unlimited
% Release". You may have to edit the file included here, or create your own
% (see the examples provided).
%
% \include{MarkOUO} % Not needed for unlimted release reports

\def\component#1{\texttt{#1}}

% ---------------------------------------------------------------------------- %
\newcommand{\systemtau}{\tensor{\tau}_{\!\text{SUPG}}}

% Added by Sonat
\usepackage{placeins}
\usepackage{array}

\newcolumntype{L}[1]{>{\raggedright\let\newline\\\arraybackslash\hspace{0pt}}m{#1}}
\newcolumntype{C}[1]{>{\centering\let\newline\\\arraybackslash\hspace{0pt}}m{#1}}
\newcolumntype{R}[1]{>{\raggedleft\let\newline\\\arraybackslash\hspace{0pt}}m{#1}}

% end added by Sonat
% ---------------------------------------------------------------------------- %
%
% Start the document
%

\begin{document}
    \sloppy
    \maketitle

    % ------------------------------------------------------------------------ %
    % An Abstract is required for SAND reports
    %
%    \begin{abstract}
%    \input abstract
%    \end{abstract}


    % ------------------------------------------------------------------------ %
    % An Acknowledgement section is optional but important, if someone made
    % contributions or helped beyond the normal part of a work assignment.
    % Use \section* since we don't want it in the table of context
    %
%    \clearpage
%    \section*{Acknowledgment}



%	The format of this report is based on information found
%	in~\cite{Sand98-0730}.


    % ------------------------------------------------------------------------ %
    % The table of contents and list of figures and tables
    % Comment out \listoffigures and \listoftables if there are no
    % figures or tables. Make sure this starts on an odd numbered page
    %
    \cleardoublepage		% TOC needs to start on an odd page
    \tableofcontents
    %\listoffigures
    %\listoftables


    % ---------------------------------------------------------------------- %
    % An optional preface or Foreword
%    \clearpage
%    \section*{Preface}
%    \addcontentsline{toc}{section}{Preface}
%	Although muggles usually have only limited experience with
%	magic, and many even dispute its existence, it is worthwhile
%	to be open minded and explore the possibilities.


    % ---------------------------------------------------------------------- %
    % An optional executive summary
    %\clearpage
    %\section*{Summary}
    %\addcontentsline{toc}{section}{Summary}
    %\input{Summary.tex}
%	Once a certain level of mistrust and skepticism has
%	been overcome, magic finds many uses in todays science



%	and engineering. In this report we explain some of the
%	fundamental spells and instruments of magic and wizardry. We
%	then conclude with a few examples on how they can be used
%	in daily activities at national Laboratories.


    % ---------------------------------------------------------------------- %
    % An optional glossary. We don't want it to be numbered
%    \clearpage
%    \section*{Nomenclature}
%    \addcontentsline{toc}{section}{Nomenclature}
%    \begin{description}
%          \item[alohomoral]
%           spell to open locked doors and containers
%          \item[leviosa]
%           spell to levitate objects
%    \item[remembrall]
%           device to alert you that you have forgotten something
%    \item[wand]
%           device to execute spells
%    \end{description}


    % ---------------------------------------------------------------------- %
    % This is where the body of the report begins; usually with an Introduction
    %
    \SANDmain		% Start the main part of the report
% these define commands but do not directly add content
\input{tsa.tex}
% these are the content sections
\input{introduction.tex}
\input{nomenclature.tex}
\input{Installation/main.tex}
\input{HowToRun.tex}
\input{ravenStructure.tex}
\input{runInfo.tex}
\input{files.tex}
\input{variablegroups.tex}
\input{ProbabilityDistributions.tex}
\input{sampler.tex}
\input{generated/optimizer.tex}
\input{database_data.tex}
\input{OutStreamSystem.tex}
\input{model.tex}
\input{functions.tex}
\input{metrics.tex}
\input{step.tex}
\section{Existing Interfaces}
\label{sec:existingInterface}
%%%%%%%%%%%%%%%%%%%%%%%%%%%
%%% Generic  INTERFACE  %%%
%%%%%%%%%%%%%%%%%%%%%%%%%%%
\input{code_interfaces/generic.tex}
%%%%%%%%%%%%%%%%%%%%%%%%%%%%%%%%%%%%%%%%%%%%%%%
%%% RAVEN  INTERFACE  (RAVEN running RAVEN) %%%
%%%%%%%%%%%%%%%%%%%%%%%%%%%%%%%%%%%%%%%%%%%%%%%
\input{code_interfaces/ravenRunningRaven.tex}
%%%%%%%%%%%%%%%%%%%%%%%%%%
%%%% RELAP5  INTERFACE %%%
%%%%%%%%%%%%%%%%%%%%%%%%%%
\input{code_interfaces/relap5.tex}
%%%%%%%%%%%%%%%%%%%%%%%%
%%% RELAP7 INTERFACE %%%
%%%%%%%%%%%%%%%%%%%%%%%%
\input{code_interfaces/relap7.tex}
%%%%%%%%%%%%%%%%%%%%%%%%%%%%%%%%%
%%%% MooseBasedApp INTERFACE  %%%
%%%%%%%%%%%%%%%%%%%%%%%%%%%%%%%%%
\input{code_interfaces/mooseApp.tex}
%%%%%%%%%%%%%%%%%%%%%%%%%%%%%%%%
%%%% OPENMODELICA INTERFACE %%%%
%%%%%%%%%%%%%%%%%%%%%%%%%%%%%%%%
\input{code_interfaces/openmodelica.tex}
%%%%%%%%%%%%%%%%%%%%%%%%%%
%%%% DYMOLA INTERFACE %%%%
%%%%%%%%%%%%%%%%%%%%%%%%%%
\input{code_interfaces/dymola.tex}
%%%%%%%%%%%%%%%%%%%%%%%%%%%%%
%%% RATTLESNAKE INTERFACE %%%
%%%%%%%%%%%%%%%%%%%%%%%%%%%%%
\input{code_interfaces/rattlesnake.tex}
%%%%%%%%%%%%%%%%%%%%%%%
%%% MAAP5 INTERFACE %%%
%%%%%%%%%%%%%%%%%%%%%%%
\input{code_interfaces/maap5.tex}
%%%%%%%%%%%%%%%%%%%%%%%%%
%%% MAMMOTH INTERFACE %%%
%%%%%%%%%%%%%%%%%%%%%%%%%
\input{code_interfaces/mammoth.tex}
%%%%%%%%%%%%%%%%%%%%%%%%
%%% MELCOR INTERFACE %%%
%%%%%%%%%%%%%%%%%%%%%%%%
\input{code_interfaces/melcor.tex}
%%%%%%%%%%%%%%%%%%%%%%%
%%% SCALE INTERFACE %%%
%%%%%%%%%%%%%%%%%%%%%%%
\input{code_interfaces/scale.tex}
%%%%%%%%%%%%%%%%%%%%%
%%% CTF INTERFACE %%%
%%%%%%%%%%%%%%%%%%%%%
\input{code_interfaces/cobraTF.tex}
%%%%%%%%%%%%%%%%%%%%%%%%%
%%% SAPHIRE INTERFACE %%%
%%%%%%%%%%%%%%%%%%%%%%%%%
\input{code_interfaces/saphire.tex}
%%%%%%%%%%%%%%%%%%%%%%%%%
%%% PHISICS INTERFACE %%%
%%%%%%%%%%%%%%%%%%%%%%%%%
\input{code_interfaces/phisics.tex}
%%%%%%%%%%%%%%%%%%%%%%%%%%%%%%%%%%%
%%% PHISICS/RELAP5-3D INTERFACE %%%
%%%%%%%%%%%%%%%%%%%%%%%%%%%%%%%%%%%
\input{code_interfaces/phisicsRelap5.tex}
%%%%%%%%%%%%%%%%%%%%%%%%%%
%%% Neutrino Interface %%%
%%%%%%%%%%%%%%%%%%%%%%%%%%
\input{code_interfaces/neutrino.tex}
%%%%%%%%%%%%%%%%%%%%%%%%%%%
%%% Prescient Interface %%%
%%%%%%%%%%%%%%%%%%%%%%%%%%%
\input{code_interfaces/prescient.tex}
%%%%%%%%%%%%%%%%%%%%%%%%%%%%%%%
%%% AccelerateCFD Interface %%%
%%%%%%%%%%%%%%%%%%%%%%%%%%%%%%%
\input{code_interfaces/accelerateCFD.tex}
%%%%%%%%%%%%%%%%%%%%%%%%%
%%% SERPENT INTERFACE %%%
%%%%%%%%%%%%%%%%%%%%%%%%%
\input{code_interfaces/serpent.tex}
%%%%%%%%%%%%%%%%%%%%%%%%%%%
%%% SIMULATE3 INTERFACE %%%
%%%%%%%%%%%%%%%%%%%%%%%%%%%
\input{code_interfaces/simulate.tex}
%%%%%%%%%%%%%%%%%%%%%%%%%%%%%
%%% ABCE INTERFACE %%%
%%%%%%%%%%%%%%%%%%%%%%%%%%%%%
\subsection{ABCE Interface}
\label{subsec:AbceInterface}

\subsubsection{General Information}
The ABCE Interface is used to run agent-based 
capacity expansion (CE) modeling for electricity market systems. 
\url{https://github.com/abce-dev/abce}

This allows inputs to be perturbed and data to be read out.

\subsubsection{Sampler}

For perturbing inputs, the sampled variable needs to be placed inside
of \$RAVEN-( )\$ like \verb'$RAVEN-(var)$'. 

\begin{lstlisting}[style=XML]
  <Samplers>
    <Grid name="grid">
      <variable name="var">
        <distribution>dist</distribution>
        <grid construction="equal" steps="1" type="CDF">0.0 1.0</grid>
      </variable>
    </Grid>
  </Samplers>
\end{lstlisting}


\subsubsection{Files}

The \xmlNode{Files} XML node has to contain all the files required to run
the ABCE model. 
For RAVEN coupled with ABCE, the \xmlNode{Files} XML node has to contain
the following files:

\begin{itemize}
  \item \textbf{settings.yml}: contains all run-specific settings for each simulation. Data specified here supersedes data specified anywhere else.
  \item \textbf{inputs/}:
  \begin{itemize}
    \item \textbf{agent\_specifications.yml}: definitions for the agents: financial parameters, starting portfolios by unit type, and mandatory retirement dates for owned units
    \item \textbf{C2N\_project\_definitions.yml}: contains project activity cost and schedule information for coal-to-nuclear projects
    \item \textbf{demand\_data.csv}: normalized peak demand levels per simulated year (used to scale the \texttt{peak\_demand} parameter)
    \item \textbf{unit\_specs.yml}: construction and operations cost and parameter data for all possible unit types in the model
    \item \textbf{inputs/ts\_data/}:
    \begin{itemize}
      \item \textbf{timeseries\_<quantity>\_hourly.csv}: hourly timeseries data for each of the following quantities in the system:
      \begin{itemize}
        \item \textbf{load}: normalized to \texttt{peak\_demand}
        \item \textbf{wind} and \textbf{solar}: wind and solar availability, normalized to the start-of-year installed capacity of each technology, respectively
        \item \textbf{reg}, \textbf{spin}, and \textbf{nspin}: ancillary service procurement requirements, in absolute terms (not scaled)
      \end{itemize}
    \end{itemize}
  \end{itemize}
\end{itemize}



The inputs in the \xmlNode{Files} section are shown below:

\begin{lstlisting}[style=XML]
    <Files>
        <Input name="settings.yml" type="">settings.yml</Input>
        <Input name="demand_data_file" type="" subDirectory="inputs">demand_data.csv</Input>
        <Input name="agent_specifications_file" type="" subDirectory="inputs">single_agent_testing.yml</Input>
        <Input name="unit_specs_data_file" type="" subDirectory="inputs">unit_specs.yml</Input>
        <Input name="C2N_project_definitions.yml" type="" subDirectory="inputs">C2N_project_definitions.yml</Input>
        <Input name="timeseries_nspin_hourly.csv" type="" subDirectory="inputs/ts_data">timeseries_nspin_hourly.csv</Input>
        <Input name="timeseries_spin_hourly.csv" type="" subDirectory="inputs/ts_data">timeseries_spin_hourly.csv</Input>
        <Input name="timeseries_reg_hourly.csv" type="" subDirectory="inputs/ts_data">timeseries_reg_hourly.csv</Input>
        <Input name="timeseries_load_hourly.csv" type="" subDirectory="inputs/ts_data">timeseries_load_hourly.csv</Input>
        <Input name="timeseries_wind_hourly.csv" type="" subDirectory="inputs/ts_data">timeseries_wind_hourly.csv</Input>
        <Input name="timeseries_pv_hourly.csv" type="" subDirectory="inputs/ts_data">timeseries_pv_hourly.csv</Input>
    </Files>
\end{lstlisting}

\subsubsection{Models}

The \xmlNode{Code} model can be used with
\xmlAttr{subType="Abce"} to run the ABCE Code Interface. The \xmlNode{executable} needs to be ABCE's executable \textbf{run.py}.

\begin{lstlisting}[style=XML]
  <Models>
    <Code name="abce" subType="Abce">
      <executable>abce/run.py</executable>
      <clargs arg="python" type="prepend" />
      <clargs arg="--settings_file" extension=".yml" type="input" delimiter="=" />
      <clargs arg="--inputs_path=inputs --verbosity=3" type="text" />
    </Code>
  </Models>
\end{lstlisting}

\subsubsection{Output Files Conversion}

The code interface reads in the \verb'outputs\ABCE_run\outputs.xlsx'. 
It will output the assets sheets as the csv file. The \texttt{asset\_id} 
variable that can be used as the \xmlNode{pivotParameter} and is a
string with the year. 

Exactly which variables will appear will vary depending on the
ABCE input files, but typical ones include \texttt{asset\_id},
\texttt{unit\_type}, \texttt{start\_pd}, \texttt{completion\_pd},
\texttt{cancellation\_pd}, \texttt{retirement\_pd},
\texttt{total\_capex}, \texttt{cap\_pmt}, and
\texttt{C2N\_reserved}.


\begin{lstlisting}[style=XML]
    <HistorySet name="grid">
      <Input>var</Input>
      <Output> agent_id, unit_type, start_pd, completion_pd, cancellation_pd, 
      retirement_pd,total_capex, cap_pmt, C2N_reserved </Output>
      <options>
        <pivotParameter>asset_id</pivotParameter>
      </options>
    </HistorySet>
\end{lstlisting}


\subsubsection{Installation of Libraries}

Installing ABCE so that RAVEN can run it requires that RAVEN and
ABCE have a superset of the libraries that they use so that both
can run.  One way to set this up is to install RAVEN, and then 
inside of the RAVEN conda environment install ABCE dependencies.
This is shown in the following listing:

\begin{lstlisting}
  #first clone raven, into a directory
  git clone git@github.com:idaholab/raven.git
  git clone git@github.com:abce-dev/abce.git
  #Switch to raven directory
  cd raven
  #install raven libraries
  ./scripts/establish_conda_env.sh --install
  #switch to using raven libraries
  source ./scripts/establish_conda_env.sh --load
  #Switch to ABCE and install
  cd ../abce/
  bash ./install.sh
  conda install -c conda-forge mesa openpyxl pytest PyYAML julia=1.8
  cd ../abce/env/
  julia
  # in julia hit the ']' key to enter package mode
  activate .
  status
  instantiate
  # exit julia with the backspace key
  exit()
\end{lstlisting}




\input{couplingAcode.tex}
\input{advanced_users_templates.tex}
\input{examplesPrimer.tex}
\section*{Document Version Information}
\input{../version.tex}


    % ---------------------------------------------------------------------- %
    % References
    %
    \clearpage
    % If hyperref is included, then \phantomsection is already defined.
    % If not, we need to define it.
    \providecommand*{\phantomsection}{}
    \phantomsection
    \addcontentsline{toc}{section}{References}
    \bibliographystyle{ieeetr}
    \bibliography{raven_user_manual}


    % ---------------------------------------------------------------------- %
    %

    % \printindex

    %\include{distribution}

\end{document}
